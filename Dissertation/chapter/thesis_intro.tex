\chapter{Introduction}\label{chap:intro}
The N-Queens problem initially started out as the eight queens puzzle. First proposed by chess composer Max Bezzel in 1848 to renowned chess players all over Germany\cite{bezzel}. The aim of the puzzle was to place eight queens on the standard eight by eight chessboard. The queen can move vertically, horizontally and diagonally in any direction on the board and thus is the most valuable chess piece on the board excluding the king. It's troublesome to find a single solution to the problem, but the real goal is to find every possible solution. The eight queens puzzle was fully solved in 1850 by Franz Nauck who successfully found all 92 solutions\cite{Nauck}. With the succession of this solution, many chess composers began to theorise about non-standard chessboards.
	
With larger sizes of boards looking to be solved, the problem transformed in the N-Queens puzzle. The rules are identical to the eight queens puzzle although the number of queens to place scales with the size of the board. As mathematicians and chess players began to solve for larger board sizes, they found that number of solutions would increase dramatically. There exists solutions for all natural numbers except for \(n = 2\) and \(n = 3\)\cite{hoffman}. As in the eight queens puzzle, finding a single non-trivial solution isn't the real goal, but finding all possible for a certain board size. 
    
Interest in computing this problem was popularised by Dijkstra in 1971 when he included in his paper "A short introduction to the art of programming"\cite{dijstra}. It became a common exercise for backtracking in many artificial intelligence and constraint programming textbooks. Backtracking can provide a complete solution to the N-Queens problem with all solutions found to a given board size. As well as being an exercise in artificial intelligence and constraint programming, the N-Queens problem also lends itself to parallel programming. 
    
The N-Queens problem is a perfectly parallel problem which makes it trivial to implement into a parallel system. This is because there is very little effort in splitting this problem into multiple parallel tasks. We can achieve this by running a solving algorithm on a partly complete board. The simplest option would be to place a queen in each position of a row or column, this would split the workload into eight distinct tasks that could be solved simultaneously. We can produce more parallel tasks by solving the problem to set depths. The exploration of solving the N-Queens problem through parallelism will be the focus of this paper.   
    
    
\section{Aims and Objectives}                                                           \label{sec:intro_aims} 
The increasing importance of parallel computing has never been clearer. With the exponential growth of processing and network speeds, parallel computing has become a necessity. Examples of parallel computing can be seen everywhere in today’s society from smartphones with multicore processors, to medical research using parallel systems to simulate protein sequence alignment\cite{li2003clustalw}. It is an industry that will keep growing alongside the computation of data as a whole. 
        
High Performance Computing (HPC) is the ability to process data and perform complex calculations at high speed and is a topic I haven't worked with. With that, the bulk of my project will be spent designing and implementing algorithms to run on parallel systems. I will be able to test my code on my personal computer as it has multiple cores. There are many existing N-Queens algorithms that have been executed on HPC's before, I intend to learn how these work and how they were implemented. 
        
Obviously the ultimate aim for this project and most N-Queen problem papers is to calculate the solution to the 28-Queens problem. Although this is an incredibly unlikely aim for myself due to the lack of time and resources I have available at my disposal. Top500 is a project that aims to maintain HPC statistics and ratings on a bi-yearly basis\cite{top500}. As of November 2021 the highest rated HPC is the Japanese supercomputer Fugaku with it's peak performance sitting at over a exoflop and some seven million cores\cite{top500fug}. Many of the current projects looking to solve the 28-Queens problem comfortably sit in the Top500 with access to these systems for prolonged periods of time. Also many of these projects have a team of academics looking to optimise every aspect of the operation. With my limited resources and access to high performance computers, alongside my set time frame, this will merely be an introduction into the world of parallel computing and thus my aims will be set accordingly. 
\begin{description}
\item[Aim 1] -- To implement a parallel N-Queens algorithm on a HPC.
\item[Aim 2] -- To measure and collect the speed and accuracy of the algorithm.
\item[Aim 3] -- Use the speed and accuracy data collected to improve and redesign the algorithm.
\item[Aim 4] -- Summarize my findings and compare the various algorithms and methods implemented.
\item[Aim 5] -- To correctly calculate the highest value of \(n\) and to compare my results to other projects.  
\end{description}
These aims were set in order to not only achieve the bare minimum of implementing a parallel algorithm on a HPC but analyse it in order to improve it. The improvement and speed of algorithms is an important part of this project so a large period of time will be dedicated to analysing and improving these algorithms. I understand with my limited resources I will be unable to produce any groundbreaking research but will put emphasis on the learning process throughout this project to help others who have similar goals. 